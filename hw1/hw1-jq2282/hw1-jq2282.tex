%\documentclass[twoside,11pt]{homework}
\documentclass{article}
\usepackage{color}
\usepackage{listings}
\usepackage{graphicx} 
\usepackage{amsmath}

\title{README for classify.py\\COMS 4705-HW1 Stance Classification}
\author{Jing Qian (jq2282@columbia.edu)}  
%\date{\today} 
\begin{document}
\maketitle

Dear Instructor, I want to use \textbf{one late day} for this homework. Thank you!

\section*{Requirements}
1. 尝试使用功能的多个子集来确定主题的最佳功能和总体模型类型。因此,这意味着您可以尝试使用具有ngrams的Naive Bayes和具有其他功能的Naive Bayes,以及支持SVM的相同功能。 
您应该使用5倍交叉验证来评估各种模型类型和特征的性能。这是您的模型选择。 

2. Code Submission: 最好的Ngram + 最好的其它模型。2 topics,所以总共提交4个模型。\\
Classify.py 包括:produce 4 models, 对每个topic,build+train, 用CV衡量上面两个最好的模型,说哪个队该topic最好,输出平均accuracy + F1+top 20 features. (是最好模型的top 20还是两个模型都要top20)
Classify.py里不需要跑feature selection,但是必须包括这个部分,并且在documentation里写如何运行。

3. 最后CV结果里不可以用自带的cv,因为会将test的feature带进去考虑

4. F1 可以随便用micro, macro, weighted avg over 5 folds.

5. 建议选择top feature  over the entire dataset using something like scikit-learn's SelectKBest.

6. SVM,kernel最重要的调节,但是调了就需要调C和tolerance. 也可以调vectorizer's minimum and maximum frequency thresholds

7. 可以Ngram不连续
%%
%\begin{equation}
%\begin{split}
%N = h & \ge \log \binom{2n}{n} \\
%	&= \log \frac{(2n)!}{(n!)^2} \\
%	&= \log \frac{2^{2n}}{\sqrt{\pi n }} \quad (\mathrm{Stirling\ Approximation}:  n! \approx \sqrt{2\pi n} (\frac{n}{e})^n) \\
%	&= 2n - o(n)
%\end{split}
%\end{equation}
%%

\section*{Model Description}



\end{document}